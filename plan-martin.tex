\documentclass[11pt, a4paper]{article}
\usepackage{color}
\usepackage{amssymb}
\parindent = 0pt
\parskip = 11pt
\begin{document}

\begin{center}
Departamento de Computaci\'on \\
Facultad de Ciencias Exactas y Naturales \\
Universidad de Buenos Aires \\[11pt]

\begin{Large}
Plan de Tesis de Licenciatura
\end{Large} \\
\end{center}

\vskip 11pt

\hrule

\vskip 11pt

T\'\i tulo tentativo: \textbf{Algoritmos basados en programaci\'on lineal entera para problemas de ruteo de buses escolares}

\hspace*{-0.35cm} \begin{tabular}{ll}
Alumno:     & Mart\'in Mongi Bad\'\i a \\
LU:         & 422/x13 \\
e-mail      & \texttt{martinmongi@gmail.com} \\[11pt]
Director:   & Javier Marenco \\
e-mail      & \texttt{jmarenco@dc.uba.ar} \\
\end{tabular}

\vskip 11pt

\hrule

\vskip 11pt

\section{Introducci\'on}

La necesidad de planificar los recorridos y las paradas de buses escolares origina una serie de problemas de optimizaci\'on combinatoria de inter\'es pr\'actico, que han sido abordados en numerosos trabajos de la literatura \cite{park10}. Existen distintos modelos para este problema, considerando variantes para la funci\'on objetivo y distintas restricciones sobre los tiempos de viaje y tiempos de caminata de los alumnos. Existen muchos trabajos que proponen t\'ecnicas heur\'\i sticas para este tipo de modelos, mientras que un n\'umero menor de trabajos se concentra en algoritmos exactos, en particular basados en t\'ecnicas de programaci\'on lineal entera. Un survey reciente sobre este problema puede encontrarse en \cite{campbell19}.

En esta tesis, se propone estudiar la variante del problema en la cual se tiene un conjunto de potenciales paradas para los buses, y el problema consiste en determinar tanto los recorridos de los buses como las paradas en las que se detendr\'a cada uno, de modo tal de llevar todos los alumnos a la escuela. Se asume una distancia m\'axima de caminata de cada alumno hasta la parada que tiene asignada, y una \'unica escuela donde se debe llevar a todos los alumnos. Se asume tambi\'en que se tiene un conjunto posible de inicios de recorridos para los buses, y que cada bus tiene una capacidad m\'axima de alumnos que puede transportar. El objetivo es minimizar la distancia total recorrida por los buses. Una variante simplificada de este problema fue presentada en \cite{sorensen16}, trabajo en el cual se propuso un modelo de programaci\'on lineal entera y se estudi\'o su performance sobre instancias generadas aleatoriamente.

\section{Propuesta de tesis}

Se propone en esta tesis \emph{estudiar la resoluci\'on de distintos modelos de programaci\'on lineal entera para el problema de ruteo de buses escolares} pro\-pues\-to en la secci\'on anterior. Las actividades planeadas para la tesis son las siguientes:
\begin{enumerate}
\item Planteo de distintos modelos de programaci\'on lineal entera para el problema. En particular, se analizar\'a la posibilidad de modelar el problema teniendo variables asociadas con cada cuadra de la ciudad (y modelando as\'\i\ expl\'\i citamente los recorridos de cada bus), en contraste con modelos que involucran un preprocesamiento de los caminos posibles entre paradas potenciales.
\item Obtenci\'on de datos reales del mapa de la Ciudad de Buenos Aires y confecci\'on de instancias realistas sobre el mapa de la ciudad.
\item An\'alisis experimental de la performance de los modelos planteados en el primer punto, sobre las instancias construidas en el segundo punto de esta propuesta.
\item Desarrollo de heur\'\i sticas primales constructivas y de redondeo ad hoc para el problema, con el objetivo de mejorar la performance de los algoritmos generales de programaci\'on lineal entera aplicados a los modelos anteriores.
\item Evaluaci\'on experimental y estimaci\'on de la mejor estrategia para resolver computacionalmente el problema sobre las instancias planteadas.
\end{enumerate}

\begin{thebibliography}{99}
\bibitem{campbell19}
W.~Ellegood, S.~Solomon, J.~North y J.~Campbell, \emph{School bus routing problem: Contemporary trends and research directions}. Omega, en prensa.

\bibitem{park10}
J.~Park y B.~Kim, \emph{The school bus routing problem: A review}. European Journal of Operational Research 202-2 (2010) 311--319.

\bibitem{sorensen16}
P.~Schittekat, M.~Sevaux y K.~S\"orensen, \emph{A Mathematical Formulation for a School Bus Routing Problem}. Proceedings of ICSSSM'06: International Conference on Service Systems and Service Management (2006) 1552--1557.
\end{thebibliography}

\end{document}

